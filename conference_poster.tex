%%%%%%%%%%%%%%%%%%%%%%%%%%%%%%%%%%%%%%%%%
% a0poster Portrait Poster
% LaTeX Template
% Version 1.0 (22/06/13)
%
% The a0poster class was created by:
% Gerlinde Kettl and Matthias Weiser (tex@kettl.de)
% 
% This template has been downloaded from:
% http://www.LaTeXTemplates.com
%
% License:
% CC BY-NC-SA 3.0 (http://creativecommons.org/licenses/by-nc-sa/3.0/)
%
%%%%%%%%%%%%%%%%%%%%%%%%%%%%%%%%%%%%%%%%%

%----------------------------------------------------------------------------------------
%	PACKAGES AND OTHER DOCUMENT CONFIGURATIONS
%----------------------------------------------------------------------------------------

\documentclass[a0,portrait]{poster}

\usepackage{multicol} % This is so we can have multiple columns of text side-by-side
\columnsep=100pt % This is the amount of white space between the columns in the poster
\columnseprule=3pt % This is the thickness of the black line between the columns in the poster

\usepackage[svgnames]{xcolor} % Specify colors by their 'svgnames', for a full list of all colors available see here: http://www.latextemplates.com/svgnames-colors

\usepackage{times} % Use the times font
%\usepackage{palatino} % Uncomment to use the Palatino font

\usepackage{graphicx} % Required for including images
\graphicspath{{figures/}} % Location of the graphics files
\usepackage{booktabs} % Top and bottom rules for table
\usepackage[font=small,labelfont=bf]{caption} % Required for specifying captions to tables and figures
\usepackage{amsfonts, amsmath, amsthm, amssymb} % For math fonts, symbols and environments
\usepackage{wrapfig} % Allows wrapping text around tables and figures

\begin{document}

%----------------------------------------------------------------------------------------
%	POSTER HEADER 
%----------------------------------------------------------------------------------------

% The header is divided into two boxes:
% The first is 75% wide and houses the title, subtitle, names, university/organization and contact information
% The second is 25% wide and houses a logo for your university/organization or a photo of you
% The widths of these boxes can be easily edited to accommodate your content as you see fit

\begin{minipage}[b]{0.6\linewidth}
\veryHuge \color{NavyBlue} \textbf{NDNCERT in Identity Manager} \color{Black}\\ % Title
\Huge\textit{A trial to implement identity manager for NDNFit using ndncert in Named Data Networking(NDN)}\\[2cm] % Subtitle
\huge \textbf{Yuyang(Peter) Rong}\\[0.5cm] % Author(s)
\huge ShanghaiTech University\\[0.4cm] % University/organization
\Large \texttt{PeterRong96@gmail.com} --- +1 (310) 307 9952\\
\end{minipage}
%
\begin{minipage}[b]{0.4\linewidth}
\includegraphics[width=\linewidth]{logo.png}\\
\end{minipage}

\vspace{1cm} % A bit of extra whitespace between the header and poster content

%----------------------------------------------------------------------------------------

\begin{multicols}{2} % This is how many columns your poster will be broken into, a portrait poster is generally split into 2 columns

%----------------------------------------------------------------------------------------
%	ABSTRACT
%----------------------------------------------------------------------------------------

\color{Navy} % Navy color for the abstract

\begin{abstract}

Sed fringilla tempus hendrerit. Vestibulum ante ipsum primis in faucibus orci luctus et ultrices posuere cubilia Curae; Etiam ut elit sit amet metus lobortis consequat sit amet in libero. Lorem ipsum dolor sit amet, consectetur adipiscing elit. Phasellus vel sem magna. Nunc at convallis urna. isus ante. Pellentesque condimentum dui. Etiam sagittis purus non tellus tempor volutpat. Donec et dui non massa tristique adipiscing. Quisque vestibulum eros eu. Phasellus imperdiet, tortor vitae congue bibendum, felis enim sagittis lorem, et volutpat ante orci sagittis mi. Morbi rutrum laoreet semper. Morbi accumsan enim nec tortor consectetur non commodo nisi sollicitudin. Proin sollicitudin. Pellentesque eget orci eros. Fusce ultricies, tellus et pellentesque fringilla, ante massa luctus libero, quis tristique purus urna nec nibh.

\end{abstract}

%----------------------------------------------------------------------------------------
%	INTRODUCTION
%----------------------------------------------------------------------------------------

\color{SaddleBrown} % SaddleBrown color for the introduction

\section*{Introduction}

Aliquam non lacus dolor, \textit{a aliquam quam} \cite{zhang2018ndnfit}. Cum sociis natoque penatibus et magnis dis parturient montes, nascetur ridiculus mus. Nulla in nibh mauris. Donec vel ligula nisi, a lacinia arcu. Sed mi dui, malesuada vel consectetur et, egestas porta nisi. Sed eleifend pharetra dolor, et dapibus est vulputate eu. \textbf{Integer faucibus elementum felis vitae fringilla.} In hac habitasse platea dictumst. Duis tristique rutrum nisl, nec vulputate elit porta ut. Donec sodales sollicitudin turpis sed convallis. Etiam mauris ligula, blandit adipiscing condimentum eu, dapibus pellentesque risus.

\textit{Aliquam auctor}, metus id ultrices porta, risus enim cursus sapien, quis iaculis sapien tortor sed odio. Mauris ante orci, euismod vitae tincidunt eu, porta ut neque. Aenean sapien est, viverra vel lacinia nec, venenatis eu nulla. Maecenas ut nunc nibh, et tempus libero. Aenean vitae risus ante. Pellentesque condimentum dui. Etiam sagittis purus non tellus tempor volutpat. Donec et dui non massa tristique adipiscing.

%----------------------------------------------------------------------------------------
%	OBJECTIVES
%----------------------------------------------------------------------------------------

\color{DarkSlateGray} % DarkSlateGray color for the rest of the content

\section*{Main Objectives}

\begin{enumerate}
\item Update NDNFit components to fit in most recent ndn-cxx library;
\item Use NDNCERT to update identity manager;
\item Explore NDN development on Android and provide experience for other Apps.
\end{enumerate}

%----------------------------------------------------------------------------------------
%	MATERIALS AND METHODS
%----------------------------------------------------------------------------------------

\section*{Packages}

%------------------------------------------------

\subsection*{NDN\cite{zhang2014named}}

%------------------------------------------------

\subsection*{NDNFit\cite{zhang2018ndnfit}}

%------------------------------------------------

\subsection*{NDN Certificate management protocol(NDNCERT)\cite{zhang2017ndncert}}
\par
	NDNCERT enables automatic certificate management in NDN. 
	In NDN, every entity should have corresponding identity (namespace) and the corresponding certificate for this namespace. 
	Moreover, entities need simple mechanisms to manage sub-identities and their certificates. 
	NDNCERT provides flexible mechanisms to request certificate from a certificate authority(CA) and, as soon as certificate is obtained, mechanisms to issue and manage certificates in the designated namespace.
\par
	In our project, we will be mainly focus on NDNCERT's client part. 
	As we would use NDNCERT client to get a certificate from CA and then act as identity manager.
\par
	For each client to get a certificate, the following should be performed:
	\begin{itemize}
		\item Select a CA server from given configuration file.
		\item \_PROBE interest should be expressed. CA would reply an available namespace. Some CAs may allow us to skip this and assume any namespace is available.
		\item \_NEW interest would be expressed to start. A asymmetric key would be generated by the user and attached in the interest to verify both user and CA's identity. Once CA got this interest, all available challenge types will be replied in a data packet.
		\item \_SELECT interest expressed with challenge type and other required information in the name of this interest. Only status information would be replied. The user should find the challenge secret through SMS, Email or other method.
		\item \_VALIDATE interest will be expressed with secret provided by the user. If the secret is correct, certificate would be granted; or the user have to re-express \_VALIDATE interest.
		\item \_DOWNLOAD interest is expressed to ask CA for certificate.
	\end{itemize}

%------------------------------------------------

\subsection*{Android \& Java Native Interface(JNI)}
\par 
	Applications like NDNFit require mobile clients. 
	In our case, we preferred Android over Apple, as Android are easier to develop and test.
	What's more, in the long run Android can be rooted to support NDN, yet Apple can't.
\subsubsection*{JNI}
	Android has to be developed using Java, yet NDNCERT and NDN are written in c++. Whether or not to use Java Native Interface has been a problem. JNI provides native access to non-Java languages, allowing us to call c/c++ functions using Java. In our first implementation we didn't use Java, instead we rewrote NDNCERT using Java. The benefits includes:
	\begin{itemize}
		\item No overheads for calling JNI.
		\item Easier to program/debug than JNI.
	\end{itemize}
\par
	However, soon we realized that they are downsizes too:
	\begin{itemize}
		\item Code base may be too hard to maintain. Same feature has to be implemented/ debugged twice.
	\end{itemize}
\par
	Therefore, currently we are using JNI. We would program JNI and fill in c++ functions that calls NDNCERT's functions.
\subsubsection*{Security}
\par
	The previous version of NDNFit was based on security V1. 
	But now we are switching to V2. 
	Different from V1 where we can take certificates to ourselves, V2 will store them in file system automatically. 
\subsubsection*{UI}
\par
	The UI would be very different from the previous version. 
	As we are requirement more inputs from the user.
	However, Android is an event driven system, that is to say, we cann't wait for user to input, instead, the system will notify us that there are new input from the user. 
	This requires us to write client code in a callback fashion.

%----------------------------------------------------------------------------------------
%	RESULTS 
%----------------------------------------------------------------------------------------

\section*{Result}


%----------------------------------------------------------------------------------------
%	CONCLUSIONS
%----------------------------------------------------------------------------------------

%\color{SaddleBrown} % SaddleBrown color for the conclusions to make them stand out

%\section*{Conclusions}

%\color{DarkSlateGray} % Set the color back to DarkSlateGray for the rest of the content

%----------------------------------------------------------------------------------------
%	FORTHCOMING RESEARCH
%----------------------------------------------------------------------------------------

\section*{Future Work}

\par 
	Identity manager is now functional, but there are more details that can be added, including identity deleting, multiple identity manage, etc. 
	There are still coding to do before this app can be used by users.
\par 
	However, identity manager, although important, is just one component of NDNFit. 
	There are other parts including DVU, DSU, etc. that need to change from security V1 to V2.
%----------------------------------------------------------------------------------------
%	REFERENCES
%----------------------------------------------------------------------------------------

\nocite{*} % Print all references regardless of whether they were cited in the poster or not
\bibliographystyle{plain} % Plain referencing style
\bibliography{reference} % Use the example bibliography file sample.bib

%----------------------------------------------------------------------------------------
%	ACKNOWLEDGEMENTS
%----------------------------------------------------------------------------------------

\section*{Acknowledgements}

I would like to thank Lixia Zhang, Arthi Padmanabhan for your help and guidance. Alex Afanasyev provided a lot of help when I was develop Android application. Finally, I would like to thanks Ren Sun and CSST for providing such great opportunity.
%----------------------------------------------------------------------------------------

\end{multicols}
\end{document}